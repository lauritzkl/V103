\section{Diskussion}

In der Auswertung wurden die Dichten der zwei Stäbe bestimmt und somit das Material
näherungsweise bestimmt.

Für den quadratischen Stab wurde Eisen als Material angenommen. Der Literaturwert
des Elastizitätsmoduls von Eisen beträgt $E = \num{196e9} \frac{N}{m^2}$ [3]. Damit weicht der
gemessene Wert um $\SI{42.9}{\percent} - \SI{54.7}{\percent} $ von dem Literaturwert ab.

Bei dem Runden Stab wurde Aluminium als Material bestimmt. Der Literaturwert ist in diesem
Fall $E = \num{70e9} \frac{N}{m^2}$ [3]. Die Abweichung der gemessenen Werte liegt also zwischen
\SI{21.1}{\percent} und \SI{28.6}{\percent}.

Es wird direkt klar, dass es eine Abweichung zu den Literaturwerten geben muss, da
die einzelnen Elastizitätsmodule, die mit unterschiedlichen Methoden bestimmt worden
sind zum Teil sehr unterschiedlich sind.
Diese Fehler könnten durch die Fehlerhafte Messung der Durchbiegung entstanden sein.
Die verwendeten Messuhren waren zum Teil etwas ungenau und die Werte waren auch
schwierig abzulesen, was zu Parallaxenfehlern geführt haben kann.
Außerdem könnte die Bestimmung des Materials auch falsch gewesen sein, da bei dem
quadratischen Stab die gemessene Dichte stark von einer Dichte eines möglichen Materials
abweicht, wodurch es zu einem systematischen Fehler gekommen sein könnte.
